%!TEX root = main.tex

\todo[inline]{Abstract}

Clarkson guidelines require the Abstract be 350 words or less.

This thesis develops new routing methods for large-scale, packet-switched data networks such as the Internet. The methods developed increase network performance by considering routing approaches that take advantage of more available network resources than do current methods. Two approaches are explored: dynamic metric and multipath routing. Dynamic metric routing provides paths that change dynamically in response to network traffic and congestion, thereby increasing network performance because data travel less congested paths. The second approach, multipath routing, provides multiple paths between nodes and allows nodes to use these paths to best increase their network performance. Nodes in this environment achieve increased performance through aggregating the resources of multiple paths.

This thesis implements and analyzes algorithms for these two routing approaches. The first approach develops hybrid-Scout, a dynamic metric routing algorithm that calculates in-dependent and selective dynamic metric paths. These two calculation properties are key to reducing routing costs and avoiding routing instabilities, two difficulties commonly expe-rienced in traditional dynamic metric routing. For the second approach, multipath routing,this thesis develops a complete multipath network that includes the following components: routing algorithms that compute multiple paths, a multipath forwarding method to ensure that data travel their specified paths, and an end-host protocol that effectively uses multiple paths.

Simulations of these two routing approaches and their components demonstrate significant improvement over traditional routing strategies. The hybrid-Scout algorithm requires 3-4 times to 1-2 orders of magnitude less routing cost compared to traditional dynamic metric routing algorithms while delivering comparable network performance. For multi-path routing, nodes using the multipath protocol fully exploit the offered paths and increase performance linearly in the additional resources provided by the multipath network. The performance improvements validate the multipath routing algorithms and the effectiveness of the proposed end-host protocol. Furthermore, this new multipath forwarding method allows multipath networks to be supported at low routing costs. This thesis demonstrates that the proposed methods to implement dynamic metric and multipath routing are efficient
and deliver significant performance improvements.